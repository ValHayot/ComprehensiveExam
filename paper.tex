\documentclass{report}

\usepackage{hyperref}
\usepackage{xcolor}

\newcommand{\note}[1]{\textcolor{green}{\textit{note}: #1}}

\begin{document}
\title{Pipeline systems and infrastructure for the efficient and open processing of Big neuroimaging Data}
\author{Valerie Hayot-Sasson}
\maketitle
\begin{abstract}
  Text of abstract.  
\end{abstract}
\tableofcontents
\chapter{Introduction}
intro here
\chapter{BigData and Open science in neuroimaging}
    \note{Likely will not be split in two separate sections}
    \section{large images}
        \begin{itemize}  
            \item BigBrain
            \item Micro CT 
            \item EM 
        \end{itemize}
    \section{large datasets}
    	\note{off the top of my head, but need to double check}
        \begin{itemize}
            \item HCP
            \item ADNI
            \item OpenFMRI
        \end{itemize}
    An effective engine to process such data requires the following characteristics
        \begin{itemize}
            \item in-memory computing
            \item data locality
            \item lazy evaluation
        \end{itemize}
        \section{open-science platforms for neuroimaging}
        		\subsection{Introduction}
		\subsection{CBRAIN}
		\subsection{OpenNEURO}
		\subsection{NeuroData}
\chapter{Workflow engines for neuroimaging data}
	\section{Definition of workflows/pipelines}
	\section{Nipype}
		\subsection{summary}
		\subsection{limitations of nipype}
	\section{PSOM}
		\subsection{summary}
		\subsection{limitations of PSOM}
	\section{SPM}
		\subsection{summary}
		\subsection{limitations of SPM}
	\section{Use of containers in Workflow engines}
		\subsection{use of containers in nipype, psoam, spm}

\chapter{Domain nonspecific BigData frameworks}
	\section{mapreduce}
		\begin{itemize}
			\item many computations can be described through map and reduce
			\item fault tolerant
		\end{itemize}
	\section{spark}
		\begin{itemize}
			\item based on map-reduce
			\item in-memory processing
			\item lazy evaluation
			\item general framework
		\end{itemize}
	\section{Use of containers with Big Data frameworks}
	\section{Hadoop Distributed file system}
		\note{maybe should go in between mapreduce and spark}
		\begin{itemize}
			\item enables data-locality in Spark/ decouples data locality logic from mapreduce\
			\item data replication
			\item fault-tolerance
		\end{itemize}
\chapter{High-Performance Computing and Grid Computing with Big Data}
	\section{Definition}
	\section{Why is use of such systems important}
	\section{Scheduling}
		\subsection{HPC schedulers: SLURM/QSUB}
		\subsection{Big Data schedulers: YARN/MESOS}
		\subsection{Multi-level scheduling w/ BigData schedulers}
\chapter{Current status of Big Data frameworks in neuroscience}
	\section{Ariel's paper}
	\section{Thunder}
\chapter{Conclusion}


%%%%%%%%%%%%%%%%%%%%%%%%%%%%%%%%%%%%%%%%%%%%%%%%%%%%%%%%%%%%%%%%%%%%%%%%%%%%%%%
%% Body of Thesis goes here.
%%%%%%%%%%%%%%%%%%%%%%%%%%%%%%%%%%%%%%%%%%%%%%%%%%%%%%%%%%%%%%%%%%%%%%%%%%%%%%%

\addcontentsline{toc}{chapter}{Bibliography}
\bibliography{bibliography}
\bibliographystyle{ieeetr}
%\bibliography{abbr,chalin,common,larch,tn}  %place your .bib files here
%\bibliographystyle{alpha}                   %the bibliography style to use

\end{document}
