\documentclass{report}                                                           
        
\usepackage{titlesec}
\titleformat{\chapter}
  {\normalfont\LARGE\bfseries}{\thechapter}{1em}{}
\titlespacing*{\chapter}{0pt}{3.5ex plus 1ex minus .2ex}{2.3ex plus .2ex}
\usepackage[left=2cm, right=2cm, top=2cm, bottom=2cm]{geometry}
\usepackage{hyperref} 
\usepackage{xcolor} 
\usepackage{ulem}                      
                                                                                 
\newcommand{\note}[1]{\textcolor{blue}{\textit{note}: #1}}                       
\newcommand{\tristan}[1]{\textcolor{red}{TG: #1}}                                
\newcommand{\weird}[1]{\uwave{#1}}  


\begin{document} 
    \title{Pipeline systems and infrastructure for the efficient
            and open processing of Big neuroimaging Data} 
    \author{Valerie Hayot-Sasson}
    \maketitle 
    
    \begin{abstract} Text of abstract.  
    \end{abstract} 
    \tableofcontents
    \chapter{Introduction}
        In recent years, the volume of neuroimaging data acquired has exceeded
        both the storage and computation capacity of a standard research 
        lab workstation. With the advancement of data sharing technologies, this 
        data has been made widely available, with the only factor impeding
        research on this data being access to infrastructure and software that 
        enables efficient processing on such infrastructure. Many research labs 
        do have access to high performance computing (HPC) clusters, however, 
        without efficient software to process this data, processing can take an 
        excessive amount of time. Ensuring efficient processing of data is 
        complex and likely beyond the knowledge of an average neuroimaging 
        researcher. As a result, frameworks for the efficient processing of 
        neuroimaging data need to be developed.

        Many neuroimaging workflows currently exist. Examples include
        Nipype, Pipeline System for Octave and Matlab (PSOM), LONI, SPM, FastR,
        Automated Analysis (AA) and Pydpiper\note{add citations here}. These 
        workflow engines aim to satify four criteria: 1) straightforward 
        workflow composition, 2) performance, 3) portability of the workflows 
        and 4) reproducibility of the analysis. While these workflow engines
        do tackle performance, it is mainly limited to minimizing computation
        time and does not consider data transfer times -- which are typically
        costly in Big Data settings. Therefore, these engines need to be adapted
        for the processing of neuroimaging data.

        In sec

        \section{Big neuroimaging Datasets}\label{datasets}
            Neuroimaging big data comes in two formats: 1) large images and 2) 
            large datasets. Large images consist of a singular images ranging 
            from 100s of gigabytes to terabytes in size. A well-known example of 
            such an image is the BigBrain \note{cite here}, a histological image 
            of the brain of a healthy 69 year-old man. At its highest resolution
            , it is 1x1x20$\mu$m or 1TB in
            size. Other examples of large images can be found in electron 
            microscopy (EM), polarized light images (PLI) and micro coherence 
            tomography (microCT) \note{need to do more research here}. Images at 
            such high resolution are important to researchers as they provide 
            insights into aspects not otherwise detectible in images at lower 
            resolutions. However, due to their size and lack of resources 
            available to process such images, research on this data remains 
            limited.

            In contrast, large datasets are consists of many images, typically 
            arising from multiple different subjects, that are too big to fit in 
            storage. The images alone being small enough for an average 
            researcher to process. Example of these datasets include the 
            UK Biobank, HCP Project, ADNI dataset,OpenfMRI, amongst many others. 
            These datasets are very large. For instance, the UK Biobank is 
            expected to exceed \note{X} Petabytes in size. As subsets
            of these datasets are manageable for the average researcher to 
            process, much analysis has gone into them. Moreover, unlike for 
            large images, toolkits have been developed for researchers to 
            process this data. However,the toolkits do not enable researchers to 
            process these entire datasets as a whole. Therefore, it is crucial 
            to develop efficient frameworks for the processing of these datasets 
            in order for researchers to harness the data in its entirety.

        \section{Platforms for  neuroscience}\label{platforms}
        \section{Infrastructure}\label{infrastructure}


    \chapter{Workflow composition}\label{workcomp}
        \section{Programming language}
        \section{Modularity}
        \section{Workflow sharing}
    \chapter{Performance}\label{performance}
        \section{Filesystems}
        \section{Scheduling}
        \section{Performance enhancement strategies}
    \chapter{Portability}\label{portability}
        \section{Containers}
    \chapter{Reproducibility}\label{reproducibility}
        \section{Provenance Tracking}
        \section{containerization}
    \chapter{Discussion/Conclusion}


    \addcontentsline{toc}
        {chapter}{Bibliography} 
        \bibliography{bibliography}
        \bibliographystyle{ieeetr}
\end{document}
