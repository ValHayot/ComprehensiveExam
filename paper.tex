\documentclass{report}

\usepackage{hyperref} \usepackage{xcolor} \usepackage{ulem}

\newcommand{\note}[1]{\textcolor{blue}{\textit{note}: #1}}
\newcommand{\tristan}[1]{\textcolor{red}{TG: #1}}
\newcommand{\weird}[1]{\uwave{#1}}

\begin{document} \title{Pipeline systems and infrastructure for the efficient
and open processing of Big neuroimaging Data} \author{Valerie Hayot-Sasson}
\maketitle \begin{abstract} Text of abstract.  \end{abstract} \tableofcontents
    \chapter{Introduction} intro here -mention hpc, data centre, cloud.
\chapter{BigData and Open science in neuroimaging} \note{Likely will not be
split in two separate sections} \section{large images} \begin{itemize}  \item
        BigBrain \item Micro CT \item EM \end{itemize} \section{large datasets}
        \note{off the top of my head, but need to double check} \begin{itemize}
        \item HCP \item UK Biobank \item ADNI \item OpenFMRI \item BIDS/BIDS
        apps \end{itemize} An effective engine to process such data requires the
        following characteristics \begin{itemize} \item in-memory computing
        \item data locality \item lazy evaluation \end{itemize}
            \section{open-science /data sharing platforms for neuroimaging} is
            open science just data sharing, computing.
            \subsection{Introduction} \subsection{CBRAIN} \subsection{OpenNEURO}
            \subsection{NeuroData} \subsection{Apache Airvata} \chapter{Workflow
            engines for neuroimaging data} \section{Definition of
            workflows/pipelines}

    Broad definition of activities - between the tool and the engine.
    \section{Nipype} \subsection{summary} Nipype\cite{nipype} is a popular
    neuroimaging pipelinening framework written in Python.  It provides users
    with uniform access to the rich ecosystem of neuroimaging software libraries
    (e.g. SPM, FSL, Freesurfer) through its \textit{Interfaces}. \weird{Should
    users want to use their own custom tool, it is also possible to create an
    interface for that}. To execute Python code in Nipype, the \textit{Function}
    interface can be used \tristan{for arbitrary code?}. The framework is also
    easy to use, and thus does not contribute to the steep learning curve new
    researchers must face.
		
        A pipeline in Nipype, consists of a data analysis. explicit
        \textit{Workflow} of connected \textit{interfaces}. The workflow is a
        directed acyclic graph that connects the outputs of one interface to be
        the inputs of another. A workflow may also consist of connected
        workflows.

                \tristan{The text below is about workflow execution while the
                text above is about workflow description, this should be a new
                paragraph.}  Workflows may be executed in parallel either
                locally or an a cluster through the use of a plugin in the
                Workflow's run function. For debugging, Nipype uses
                \textit{graphviz} to generate a static graph representing the
                Nodes and their relationships. \tristan{debugging might be a
                third theme, in addition to description and execution.
                Provenance might go in debugging.}
		
         Each \textit{interface} within the workflow must be contained within a
         \textit{Node} or a \textit{MapNode} object, the MapNode class being a
         subclass of the Node class. Wrapping an interface in such objects
         ensures that interfaces are executed within a uniquely named
         directories, which in turn enables provenance tracking. Node objects
         also hash inputs, provide the ability to iterate over inputs and cache
         results. Hash inputs are useful in the case of Workflow recomputation,
         where Nipype will only recompute Nodes whose inputs have changed since
         the previous run.
		
        Inspired by the MapReduce paradigm, the MapNode copies the interface to
        each input and executes it independently. \tristan{How are inputs
        defined/combined?} A reduce is subsequently performed by the MapNode to
        return the output in the form of a list. This differs from a Node object
        which can only execute the interface on a single input at a time.
		
        To enable execution of different parameters on the input, Nodes have a
        property known as \textit{Iterables}. Similarly to a MapNode, a copy of
        the \textit{interface} will be made for each input. However, with
        \textit{iterables}, a copy of each dependent node will also be made.
		 
        \subsection{Limitations} Nipype, although meeting the various needs of
        researchers in Neuroinformatics \tristan{vague}, was not designed with
        the processing of Big Data in mind. Moreover, its performance relative
        to current Big Data frameworks remains unknown \tristan{You could refer
        to your MS thesis here}.  \note{nipype performs a lot of disk I/O.
        Recomputation of workflows vs RDD caching}
		
    \section{PSOM} \subsection{summary} While Python is a popular programming
    language amongst neuroinformaticians, it is not the only one that is
    commonly used. To address the needs of the Matlab/Octave community, PSOM
    (Pipeline System for Octave and Matlab) was developed. Similarly to Nipype,
    PSOM is easy-to-use, can run in distributed environments and can interface
    with tools written in other languages.

        \subsection{pipeline description} A pipeline in PSOM, is a series of
        connected jobs, which can be represented as a (static?) directed acyclic
        graph. The pipeline itself is implemented using the Matlab/Octave data
        type. Each job within the pipeline is assigned a unique name. There is
        only one mandatory field in the job description (command). The remaining
        four optional fields include the list input files (files\_in), the list
        output files (files\_out), the list of files to be deleted after
        execution (files\_clean) and any Octave or Matlab datatype containing
        the remaining parameters necessary for job execution (opt).  Jobs in
        PSOM run in a protected environment where they only have access to the
        parameters that are passed. 

        In order to determine whether the opt field was correctly set, the
        generic function \textit{psom\_struct\_defaults} may be applied. This
        function will set the default values, ensure all mandatory fields were
        supplied and issue warnings for all unidentified attributes.

        To define modules in PSOM, a special Octave/Matlab function type, known
        as "bricks" is used. Bricks employ the same parameters as jobs but lack
        the file\_clean parameter. All bricks contain an opt.flag\_test boolean
        flag, which, if set to true, allows the user to perform a dry-run of the
        brick. A dry-run of a brick consists of updating the default parameters
        and validating that the brick can successfully execute with the
        parameters provided. Bricks also have an option to specify the output
        folder (opt.output\_folder) for brick outputs to be stored.

        The pipeline generator is an easy way to create complete pipelines with
        minimal information provided. A pipeline generator may accept at most
        the file\_in and opt arguments. It then builds the pipeline based on
        this information and the default parameters specified in each brick
        applied by the pipeline generator.

        \subsection{pipeline execution}
        
        To execute a pipeline, the user must provide the pipeline structure and
        the pipeline configuration options. The latter consists of the log
        directory, which must be provided, and the pipepline execution modes
        (e.g., session, background, batch, qsub, msub), in which the default
        mode is \textit{background}. Pipeline execution is achieved through the
        execution of the \textit{psom\_run\_pipeline} command. 

        Pipeline execution is broken down into three distinct modules: 1)
        initialization stage, 2) pipeline manager and 3) job manager. The
        initialization stage performs some checks to ensure that the pipeline
        can execute successfully (e.g., verify that pipeline is a DAG, check
        that output files are generated only once, check that all input files
        not generated by the pipeline exist on disk, delete any already existing
        output files), as well as performance checks (e.g.  do not recompute
        pipeline steps that have already been executed, create necessary output
        folders to avoid creating them for each individual job). The
        initialization manager can determine which jobs need to be executed by
        comparing the current pipeline jobs with the details of those previously
        executed stored in the logs folder. The jobs that have been executed can
        have the following states: 1) none, 2) finished and 3) failed. The
        initialization manager stage the jobs with a \textit{none} or
        \textit{failed} status for recomputation. For jobs whose description has
        changed since the previous \textit{finished} execution, the
        initialization will also stage it for recomputation. Should the user
        want to reexecution a previously executed \textit{finished} job, the
        user may force reexecution by setting \textit{opt\_pipe.restart}
        parameter to include all jobs to be reexecuted.

        The pipeline manager is responsible for monitoring the state of the
        pipeline and submitting jobs for execution. Its duties begin after the
        initialization manager has completed execution. Jobs can be submitted by
        the pipeline manager once all their dependencies have been satisfied.
        The dependencies of a job are designated as met if there are none or
        when its parent jobs within the dependency graph are tagged as
        \textit{finished}. In addition, a job is considered to be successfully
        completely when all the dependencies of its children jobs have been
        satisfied. Depending on the pipeline configuration, there may be a limit
        to how many jobs can run in parallel. Once a job has finished executing,
        an empty tag file is created with a filename containing the final state
        in the logs folder. The pipeline manager continuously parses through the
        logs folder in search of these state files to delete and update the
        history file with the job state. Both the pipeline manager and the jobs
        can be executed within the current session, locally in a independent
        session, or in an independent session on a cluster

        
        Every time a job is submitted by the pipeline manager for execution, a
        job manager is started. It is a Matlab function responsible for creating
        the job profile and generating the job-specific logs and the job exit
        tags. When the job terminates, the job manager will verify that all
        output has been correctly generated and assigns a job tagged based on
        this verification that will be reported back to the pipeline manager.

        \subsection{debugging and provenance}

        The information stored in the logs folder is complete with details
        allowing for both debugging and provenance. The logs folder contains six
        files: 1) PIPE\_history.txt, 2) PIPE\_jobs.mat, 3) PIPE\_status.mat, 4)
        PIPE\_logs.mat, 5) PIPE\_profile.mat and 6) PIPE.mat. PIPE\_history.txt
        includes the execution history of the pipeline manager. PIPE\_jobs.mat
        and PIPE\_profile.mat are both Matlab/Octave files containing each
        executed job as a Matlab variable. They differ in that PIPE\_jobs.mat
        contains the latest executed jobs and can be used to reexecute the
        pipeline at any moment in time and PIPE\_profile.mat contains the
        execution time of each job \note{i think both can be used to reexecute
        pipeline from scratch. Also unclear whether PIPE\_profile.mat contains
        more than just latest jobs}. It is also possible to access any of the
        previously used parameters through PIPE\_jobs.mat. PIPE\_status.mat
        stores the status of each job as a string variable, whereas
        PIPE\_log.mat stores each job's log as a string variable. Job logs, as
        generated by the log manager, consist of a text copy of the execution,
        in addition to the user name, date, time and operating system used to
        execute the job. Finally, all configuration details are stored in
        PIPE.mat. 
       
        \subsection{limitations of PSOM} \section{SPM and matlabbatch}
        \note{This section may not make much sense. Have not really yet fully
        understood matlabbatch} \subsection{summary} SPM is a very popular
        neuroimaging toolkit, and as a result, neuroimaging pipeline engines
        such as Nipype provide interfaces for it. SPM is also equipped with its
        own neuroimaging pipeline engine known as \textit{matlabbatch}.
        Similarly to PSOM and as its name suggest, it is written in Matlab.

        \subsection{pipeline description} Pipelines in matlabbatch consist of a
        series of connected data processing steps known as "modules" \note{DAG??
        probably but not explicitly listed anywhere}. All SPM modules, helper
        modules and their respective dependencies can be defined using SPM's GUI
        \note{can probably also be done through command-line}.

        For any matlabbatch job, it is possible to create a skeleton of it. A
        skeleton is a Matlab script containing a commented list of necessary
        inputs at the top and all code necessary to initialize and run the
        script. Before running, the skeleton will need to be configured with the
        input names and other necessary parameters.

        All configuration options can be set/changed by modifying the
        \textit{cfg\_$\langle type \rangle$ } files. 

        Defaults can be set within the configuration files, through a default
        file, or in a \textit{.def} field for leaf nodes \note{terminal
        jobs???}. The latter updates at the start of the new job execution
        whereas the former are configured at the beginning of the
        SPM/matlabbatch startup.


        Virtual outputs represent the dependencies between jobs \note{unsure}.
        They contain three paramaters of which only one is optional: sname,
        src\_output, tgt\_spec. sname is the display name that will be stored in
        the dependecies list, whereas src\_output is a subscript reference used
        to refer to the output that is produced at runtime. The optional
        parameter, tgt\_spec, is a description of the kind of inputs the outputs
        produced can be a dependency for. The virtual output callbacks are
        evaluated when cfg\_(ex)branch, cfg\_choice and cfg\_repeat have the
        correct number of child nodes within all their in-tree nodes. 

        \subsection{pipeline execution} Pipeline execution can be achieved
        either through the provided GUI or by the command-line interface. The
        GUI, however, requires significant user interaction, as the user must
        repeat the "Runs" command as many times as necessary to reexecute the
        command on multiple data. The CLI, however, avoids this with the help of
        its \textit{spm\_jobman} interface.

        The first step to running a batch pipeline is to perform the
        initialization. This may include configuring the
        \textit{spm\_defaults.m} code to run in command line and executing the
        \textit{spm(’defaults’, MODALITY)} (where MODALITY represents the
        modality used in the pipeline) and \textit{spm\_jobman(’initcfg’)}.

        After the intialization is complete, a job may be executed using the
        \textit{spm\_jobman(’run’, job[, input1, input2 ...])} command. This
        command requires the job names as well as its necessary inputs specified
        in the correct order. The job can be specified as a variable, the name
        of a script that will generate a job variable or a cell list containing
        a mixture of scripts and job variables. 
         
        \subsection{debugging and provenance} All batch scripts can be saved as
        a Matlab scripts. These scripts not only contain the code that was
        executed, but also all input parameters. Should any script fail in the
        process of executing, an error will be printed to the Matlab console. If
        the user is using the GUI, the error output may also be saved to a
        Matlab file.
    
        \subsection{limitations of SPM}
    

    \section{FastR} \subsection{summary} FastR is another Python-based
    neuroimaging pipeline framework that provides added features such including
    access to different versions of the same command-line tool \note{criticizes
    nipype for not having such a feature, but nipype can access boutiques
    tools...so...} and compatibility with various possible filesystems. In
    addition to its added features, FastR also guarantees a simple-to-use
    framework for building pipelines written in any languages, has built-in
    provenance support, provides the seamless ability to parallelize pipelines
    with the use of plugins and avoids the numerical instability issue with the
    simple integration of containerized pipelines. Similary to many other
    pipeline framework, it also employs a DAG to describe data flow.
 
        \subsection{Pipeline description}

        A pipeline in FastR is a \textit{Network} of connected \textit{Nodes} in
        which the directed edges represent the flow of data.  Each \textit{node}
        contains a JSON or XML template known as a \textit{Tool}. 

\textit{Tools} describe the details of the underlying application that is to be
executed at the node. Tools consist of three components: general metadata,
target and interface.  The general metadata provides details on the Tools such
as name, version, authors, etc, whereas the target specifies details on the
execution environment.  The interface, on the other hand, provides details on
the inputs and outputs of the tool. 

A JSON schema is required to validate the python object generated by the Tool.
Fastr currently provides two schemas for object validation: the \textit{Tool}
schema and the \textit{FastrInterface} schema. It is, however, possible to add a
custom schema to FastR and define an \textit{Interface} for it. 

The \textit{Interface} class is responsible for executing a call to the command
line tool.  It dictates the tags that must be defined within the interface tag
of the \textit{Tool}. For instance, the generic interface class requires details
on the input and output's id, cardinality (number of input values) and DataType
(a python object denoting the details of the Input or Output's data type).
DataTypes can also be grouped, allowing an inputs containing various files types
to be inputted to applications which can support the file types.  Verification
is done by FastR to determine if the linked Input and Output data types are
compatible.

Input samples can consist of one or more values, in which the output is also a
sample consisting of the same number of values. During the instances in which it
is desirable to apply the same task to multiple inputs, a multidimensional array
of samples, known as a collection, can be supplied to a \textit{Node} for
processing. If there are multiple input samples, they can be passed to a node in
pairwise behaviour (e.g. 1-1) or cross-product behaviour (e.g. nxn) \textit{need
to better explain this}.  

Output samples may need to be expanded (i.e. a single sample of values becomes
separate samples) or collapsed (i.e. multiple samples become a single sample of
values).  FastR provides data link flow directives for such operations.

In FastR, there are four types of \textit{Nodes}: normal Nodes, Sink Nodes,
Source Nodes and Constant Nodes. Data enters the network through Source or
Constant Nodes and exit through Sink Nodes. Constant nodes differ from Source
Nodes in that their input data are defined by the network \note{Not sure if
saying this correctly}. Normal nodes are responsible for processing the data.
Nodes are connected to each other through links that highlight the dependencies
and form the \textit{Network}. Once the \textit{Network} is defined, it can be
executed.

The data type of the Source Node and Sink Node must be defined in the
\textit{Network} definition.  The data is define at runtime by evaluation of the
uniform resource identifier (URI). Multiple URIs can be defined through the use
of wildcards and searches. Plugins are used to read and write data to different
filesystems specified in the URI scheme. Currently, FastR supports virtual file
system (VSF), csv files and XNAT. Plugins to support other filesystems can
easily be added to FastR. In addition, plugins that can only read, write, or
perform searches can also be created.   


        \subsection{Pipeline execution} To commence pipeline execution, the
        \textit{Network}'s execution function is called on the Submission host.
        FastR then analyses the network and its dependencies and breaks down the
        Network into chunks that can be further processed. From the chunks, the
        execution order of the \textit{Nodes} is determined. 

During \textit{Node} execution, a job is created for each input/output
combination. Jobs are then executed on the execution host (i.e. local, cluster,
cloud), as determined by the execution engine plugin. FastR currently only
offers support for local and cluster plugins, however can easily be extended to
support other plugins. 

While the job is executing, input URIs are translated to their actual
path/values, the execution environment is set up by the \textit{Tool} and the
\textit{Tool} command is executed by the \textit{Interface}. When the
\textit{Interface} returns the results, the output data is validated and
translated to URI format. 

After job completion, the results are returned to the Submission Host which
reads the output and updates the network accordingly. Should a chunk be
completed, the next chunk will be processed with the updated information, until
all chunks have been processed and network execution is complete.

        \subsection{Debugging and Provenance} In FastR, a provenance document is
        generated for each resulting file generated in the Prov-N format. Data
        recorded in the provenance documents include \textit{Tool} versions,
        input/output paths/values and their respective checksums, start and end
        time of execution for each \textit{job}, stdout and stderr logs, the end
        status and the complete environment description.

    Similarly to Nipype, the FastR network can be visualized using graphviz.

         

    \section{LONI} \subsection{Summary} The LONI pipeline processing environment
    facilitates the development, sharing and execution of pipelines. Workflow
    Activities, known as \textit{Modules} in LONI, are described using
    Extensible Markup Language (XML), which can easily shared between user
    groups. LONI runs on the Java platform, allowing it to be platform
    independent and has support DRMAA clusters and symmetric multiprocessing
    (SMP) systems. In order to capture provenance information, LONI is also
    equipped with a provenance manager.
  
        \subsection{Pipeline description} LONI uses the dataflow programming
        model to represent the pipeline. Data flows between \textit{Modules}
        through \textit{data pipes}. Sub-pipelines are known and
        \textit{Pipelets}. Modules are defined by the input arguments/streams
        and output arguments/stream. Arguments in LONI may be specific
        filetypes, flags or values listed in the order in which the application
        requires. The provided arguments may be required or optional and can
        also be duplicated, if necessary. File inputs in the form of URLs are
        accepted by LONI and will be downloaded at runtime. List of inputs may
        also be provided to process multiple inputs with copies of the same
        module in parallel (batch processing). Conversely, it is also possible
        to group list of inputs so as to process them all by the single module
        \note{Need to express this better}. Input/Output arguments may be added
        to the table of global variables for easy access throughout the
        pipeline. It is possible to apply variable modifiers to the variables
        stored within the table.

Intermediary outputs as well as the file output are treated as temporary by
LONI. They are scheduled for deletion before a new pipeline is executed. It is,
however, possible to preserve these files. 
  
        \subsection{Pipeline execution} As a module completes it execution, its
        output gets transmitted through the dataflow and the dependent modules
        are signaled that their input data has arrived.  Modules who have met
        all their dependency requirements are ready for execution.  Should a
        module have no dependecies, it can be executed immediately. Modules
        ready to be executed are placed in a FIFO launch queue to avoid
        straining the platform. 

       LONI offers a built in client/server strategy that uses Java's remote
       method invocation (RMI) libraries to communicate between servers. This
       enables modules, pipelets or pipelines to be executed on servers
       different from the client. It also enables full pipelines to consist of
       connected modules and pipelets located on different servers. It is
       favorable to have such a configuration in LONI, as the tools required for
       processing may not be available on client server, but located in another.
       The client is responsible for dictating all file transfers and module
       launches that are to occur. The client authenticates with the servers
       through a one-way hash. Only files produced by the pipeline are
       accessible to the client and only temporary files created by the server
       may be written to the server.

        \subsection{Debugging and provenance} For each dataset processed in a
        LONI pipeline, there exists a corresponding provenance (.prov) XML file.
        The \textit{Provenance Editor}, a standalone platform-independent
        application, reads the metadata from the image headers, extracts
        relevant provenance information from them and creates the provenance
        files.  This data can be modified by the user in the to correct
        inaccuracies or append additional details before being saved. During
        pipeline execution, the provenance file will be edited by the LONI
        pipeline to include details on each processing step, the environment in
        which the data was processed and tool binaries used. 

    \section{Automatic Analysis} \subsection{Summary} Automatic analysis (aa) is
    another Matlab-based modular workflow engine, having started of as a
    workflow engine for SPM. User may define their pipeline through a Matlab
    user script and specify the tasks in addition to their input and output
    types through an XML module interface. Modules used in pipelines must be
    integrated into aa which can be easily accomplished through XML interfaces.
    aa workflows can be executed locally, or in parallel through qsub or condor.

        \subsection{Pipeline description} aa pipelines have two main components:
        1) the Matlab user script and 2) the XML tasklist. The user script is a
        script that defines which data will be processed, any necessary
        parameter, and how it will be processed. aa provides an example user
        script, which can be modified to meet user's needs. The entire pipeline
        in aa is defined in a single structure, known as aap, which makes the
        task of record-keeping simple.

        The XML tasklist is an ordered list of all the modules that will be
        applied to the data, that is read by the user script. This list does not
        specify any dependencies as those dependendecies can be inferred by the
        pipeline through analysis of the module interfaces. Modules can be
        separated into two sections: initialization and main. Modules listed in
        the initialization section are to be executed for every task.
        Conversely, those modules liste in the main section are only executed
        once for every task, unless otherwise specified.


        In order to ensure that the correct output data is being operated on,
        each task is provided with its own unique folder. The aa engine will
        transmit the necessary input data to the folder before task execution
        and also acquiring the outputs after task completion. Output folders are
        numbered such that if a given module is invoked multiple times, each of
        those folders are assigned an incremental ordering corresponding to the
        order in which they have been executed \note{sentence might not make
        sense}. Should all data be required for each task in the tasklist, all
        the data will be copied to each task folder.


    Similarly to a pipeline, each module is comprised of an XML interface and a
    Matlab source file. A module's XML file consist of specifications on
    required input and output modalities, as well as the desired domain. One of
    the most important properties include the domain at which the module
    operates. While there exist mainy domains for modules and new ones can
    easily be created and integrated by the user, the most prominently used ones
    include: "study", "subject", "session". As their name suggests, a module
    with the domain "study" will only be executed once for each time it appears
    in the tasklist, whereas "subject" executes the module for each subject
    within the task and "session" executes once for each session within a
    subject. 
 
    In addition to module domains, another important module property is the
    input and output data stream types. Data stream types denote what input and
    output data type each module produces. Commonly used data stream types
    include "epi", "structural" and "dicom\_header".

    Optional fields specified in a module include qsub, for estimates of
    requirements for parallel processing, and \textit{performanceoutput}, for
    garbage collection.  

   Analysis parameters can easily be configured in aa. Any variable defined by
   the \textit{aaprecipe} command may be modified. Parameters can also be
   defined for an instance of a module within the XML tasklist. Alternatively,
   it is also possible to create an XML file that inherits parameters from
   another XML file and overrides some or all of the parameters.

   To set defaults and site-specific configurations, it is possible to either
   create a local config file or by creating another parameter default file
   which inherits the original one.   

    aa also supports the creation of branched tasklist. This allows a section of
    the pipeline steps to undergo different tasks or the same tasks in a
    different order in one pipeline rather than having to create separate
    pipelines.

        \subsection{Pipeline execution} Once the definition of the user-script,
        XML tasklist and modules is complete, the pipeline is ready for
        execution. Dependencies in aa are established based on input and output
        data stream types. The aa engine attempts to match the last task
        executed with a corresponding data stream type to the current type. It
        is, however, possible to specify which module provides the input to the
        current module by using fully qualified stream references.

        During execution, a map of all the modules that need to be executed and
        their respective dependencies is generated. Modules that have completed
        contain a flag file indicating their completion within their instance's
        root directory. Such modules will not be re-executed by the engine.
        Additionally, if it is required that a previous module is re-executed,
        the done flag is removed from all dependent modules.  

        Should part of the processing be performed on a different server it is
        possible to connect the pipelines together through an analysis script.
        The aa engine will be reponsible for matching the data streams between
        the servers in order to avoid the copying of data and modules.

        \subsection{Debugging and provenance} In order to retrace the steps of
        previous executions, aa keeps a detailed record of the events that
        occurred. These detailes are stored in a Matlab object \note{similarly
        to PSOM}, which permits reexcution of a previous run or extraction of
        specific parameter values.

        Additionally, to verify the correctness of the results produced, aa
        implements tools for quality control and comes with a dedicated module
        for low-level quality control.
    

          
    \section{Pydiper} \subsection{Summary}

        Pydiper is a neuroimaging registration workflow engine implemented in
        Python. This differs from other workflow engines in that it is specific
        to the process of image registration, but can, however, be extended for
        use with any analysis process. Its main features include being easy to
        use, particularly for non-developpers, undertaking a modular approach,
        giving the ability to access output files at any stage, and eliminating
        duplicate stages. Pydiper also provides the added advantage of being
        equipped with four commonly used registration pipelines, such that
        unexperienced researchers can use such pipelines out-of-the-box. 

        \subsection{Pipeline description} Pydiper workflows can be broken down
        into three components: atoms, modules and applications. An application
        is the command line application that encapsulates the pipeline which
        consists of modules and atoms. Atoms are wrappers for individual
        operations, typically frquently used MINC tools. Modules are reponsible
        for linking atoms together.

        Pydpiper provides three main classes to use to develop pipelines. These
        include the \textit{PipelineStage}, \textit{CmdStage} and
        \textit{Pipeline} classes. The \textit{PipelineStage} is the base class
        wrapper for all pipeline tools.  Pipeline tools can be written in Python
        or can be a command-line application.  For the tools that are
        command-line applications, the \textit{CmdStage} class, a child class of
        \textit{PipelineStage}, must be used. Input to \textit{CmdStage} is in a
        adequately parsed array format containing the specific command-line tool
        and its arguments. Reusable subclasses of the \textit{CmdStage} class
        can be created, or \textit{CmdStage} can be called directly.

        To combine pipelines, the \textit{AddPipeline} function of the
        \textit{Pipeline} class may be used.

        The base class common to all Pydpiper applications is
        \textit{AbstractApplication}.  Classes that inherit from
        \textit{AbstractApplication} are command-line tools, with their options
        set by the \textit{AbstractApplication} class. The class is also
        responsible for initializing or resuming the pipeline, setting up a
        logger, and launching the pipeline daemon.

        Pydpiper atoms inherit from the \textit{CmdStage} class in order to wrap
        the MINC Tools.  They require as input a MINC file, and may additionally
        also require a target MINC file as second argument. These arguments can
        be in the form of a string filename or a file handler, however, the same
        format must be used for the two. An atom may also accepts various
        optional arguments. Using the inputs provided, the atom will put
        together the command to be executed. 
        
        \subsection{Pipeline execution}
        
        In Pydpiper, there are two main classes,\textit{Pipeline} and
        \textit{PipelineExecutor}, that monitor and manage pipeline execution.
        The \textit{Pipeline} class's main responsibility is to infer
        dependencies between pipeline stages from their inputs and outputs. It
        also manages a queue of runnable stages (stages in which all their
        dependencies have been met), ensuring that any runnable stage is placed
        in the queue. It also keeps track of the status of the runnable stages
        (running, finished, or failed). Should an already existing stage be
        added to the pipeline, it is automatically skipped by the
        \textit{Pipeline} class. 

        As stages may be added to incomplete pipelines, Pydpiper provides the
        functionality of resuming pipelines. This is achieved through a pickling
        mechanism implemented by the \textit{Pipeline} class. The pickled
        components include the directed acyclic graph, the array of pipeline
        stages, the stage counter, the hashes of all stages and each of their
        output files, and the status of each stage.

        The \textit{PipelineExecutor} class, on the other hand, is responsible
        for the efficient processing of the pipeline through parallelization.
        Pydpiper's architecture mimics that of client/server, in which the
        server is the pipeline daemon started by the command-line executable and
        the executors undertaking the role of the clients. The pipeline's
        executors each consist of a thread pool in which each thread is
        responsible for running a single pipeline stage from the queue of
        runnable stages. To obtain a stage for execution, the executor polls the
        server to obtain the next available stage. Should the executor have
        enough memory and processors to execute the stage, it will be executed
        by the thread and release memory and processors and repoll the server
        once stage has completed.  Otherwise, the executors will sleep for
        certain duration and repoll the server again.  Should there be any
        failed stages, the pipeline will be shut down after all possible stages
        have finished executing. If additional executors are required by the
        pipeline, they may be added at any time through the command-line.
        Details related to the number of threads, number of threads per executor
        and the amount of memory can be specified through the command-line
        options. 

    
        \subsection{Debugging and provenance}

        As mentioned previously, Pydpiper uses a pickling mechanism to gather
        important pipeline details during execution in order to allow for the
        pipeline to be resumed should any stages have failed.  In addition, to
        keep track pipeline stage interdependencies in addition to file
        versioning, Pydpiper utilizes the helper class
        \textit{RegistrationPipeFH} and its parent \textit{RegistrationFHBase}
        to create file handlers of the files.

        \note{from github wiki} For debugging, a Pydpiper pipeline creats its
        own log file and executors typically also each create their own log file
        that are stored in a \textit{logs/} subdirectory. Verbosity of logging
        information can be managed through the setting of the PYRO\_LOGLEVEL
        environment variable. In addition, individual stage stdout/stderr is
        also redirected to a log file.

        To verify the status of an executing pipeline, the
        check\_pipeline\_status.py may be executed.
 

\chapter{Other domains and general scientific worflow engines} \section{Taverna}
Taverna is an open-source bioinformatics workflow engine, designed to allow
users to combine distributed web services, other service types (e.g. BioMart
queries or R-scripts) and local scripts into complex analysis pipelines. The
main goal of Taverna is to give a unified access to the plethora of services
already existing on the web without requiring the user to download the services
locally or to have knowledge on all of the platforms and programming languages
the services are hosted under.

    

        \subsection{Pipeline description} Workflows in Taverna are built by
        combining web services, local scripts and other services together. Web
        services compatible with Taverna must comply to the Web Service
        Description Language (WSDL) using the Simple Object Access Protocol
        (SOAP protocol).  These services can be added to the BioCatalogue or the
        Biodiveristy catalogue for sharing and reuse of the web services. 

        Workflows can be built in the Taverna workbench. Taverna, through the
        help of the BioCatalogue and the Biodiversity Catalogue is able to
        provide the user with a variety of webservices to combine into workflows
        in a drag-and-drop fashion. Workflows may also be composed of
        sub-workflows. As service inputs and outputs may not match between
        services, shims are provided.  Similarly to the services, the
        user-generated workflows can be uploaded the myExperiment repository to
        be shared with other user.
        
        \subsection{Pipeline execution} The workflows can be executed remotely
        utilizing the Taverna Server. The Taverna Server is a an application
        that provides the full library of possible workflows through
        webservices. It has an interface, known as the Taverna Player, that
        enables workflow execution to occur through a web-interface or through
        third-party clients. The Taverna Server, however, is limited in that it
        can only execute workflows available on the Server. Taverna Lite is an
        alternative to the Taverna Server which allows users to upload and
        execute workflows from myExperiment or elsewhere.   
        
        \subsection{Debugging and provenance} Provenance capture is achieved
        through the Taverna Provenance Suite. Taverna does not capture
        provenance of the workflow definition, but generates a new UUID after
        each definition change. Workflow run provenance captures information on
        "individual processor iterations" and their respective inputs and
        outputs and stores this information in a database. The results can then
        be exported as a workflow trace or as a PROV-O RDF graph using the
        Taverna-PROV Plugin. The model format used for exported provenance uses
        both the W3C PROV model and Open Provenance Model format.
        
    \section{Pegasus} Pegasus is a general-purpose scientific workflow engine
    that specializes in mapping abstract descriptions of workflows to resources
    for efficient processing.  Similarly to other workflows, a Pegasus workflow
    can be represented by a DAG. This DAG is described by the user in DAX format
    (Directed Acyclic graph in XML).  As the DAX abstract workflow is
    resource-independent, Pegasus can convert the abstract workflow into an
    execution workflow that is catered to the resources available.        

        \subsection{Pipeline description} Within the DAX, the nodes are the
        tasks and the execution order is represented by the edges.  In the
        instance of nested graphs, the nodes in the DAX may also represent a
        DAX. Pipeline inputs and outputs, as well as pipeline executables, are
        represented by logical identifiers.

         Various APIs are provided by Pegasus in order for the user to generate
         the DAX using their preferred language of choice. Available APIs
         include Perl, Python and Java.  \subsection{Pipeline execution} With
         the effecient execution of pipelines being Pegasus' main focus, there
         exist various components that ensure optimal processing. The Mapper's
         responsability is to map abstract workflow tasks to available
         resources. In order for the Mapper to successfully convert an abstract
         workflow into an execution workflow, it requires details on the
         execution environment. The Mapper will also query various catalogs to
         determine the location of executables and input and output files. These
         catalogs can be either file-based or in database format. With all the
         necessary information provided, the Mapper is equipped to convert
         abstract tasks into executable tasks. Data nodes are also added for
         input and output staging. Should the data required for a task already
         exist in the catalog, the task and all its parents will be removed by
         the Mapper from the workflow. This functionality is optional, and in
         some instances, it may be found that recomputation is preferable. If
         the executables required by the task are not pre-installed on the node,
         the Mapper will query the \textit{Transformation Catalog} for the
         stageable binaries.

        The default, Pegasus relies on the HTCondor framework as both the
        workflow engine and scheduler. The executable workflow, using HTCondor
        DAGMan, consists of nodes representing jobs.  These jobs each have a
        corresponding job submit file that details how they are meant to be
        executed. A job that is ready for execution is released by the DAG and
        placed into a Condor queue that is managed by the HTCondor Schedd
        daemon. The default scheduling is done using a local HTCondor pool, but
        can also be achieved through built-in interfaces with remote
        job-management protocols, or through SSH to directly submit jobs to
        remote resource managers.

        Using HTCondor, Pegasus is able to achieve portability. Pegasus is
        compatible with local, HPC, distributed and cloud execution
        environments. As distributed environments may not employ a shared
        filesystem, it is recommended to use HTCondor glideins to ensure that
        tasks are mapped to nodes where the resources are located. Should a user
        running their pipeline locally without HTCondor installed, it is
        possible through \textit{submit host}. Using this method, however,
        executes the pipeline in serial order. Should the user wish to exploit
        the pipeline's parallelism, HTCondor is required. Pegasus pipelines are
        not limited to a single execution environment, and can combine multiple
        environments to select the best resource configuration to execute each
        job.

        In order to manage tasks and their dependencies on remote resources,
        three different execution engines have been developed. The first and
        simplest execution engine is the \textit{Pegasus Cluster}, which is
        compatible with all the execution environments. It runs on a
        single-thread, meaning that it can only executed a single task at a time
        using an ordered-list of tasks.  The \textit{Pegasus Cluster} is
        particularly useful for the efficient execution of short tasks and the
        reduction of data transfers. 
        
        For execution environments lacking a shared filesystem, the
        \textit{PegasusLite} execution engine can be used. In order for the
        Pegasus to execute in such environments, the Mapper first adds data
        transfer nodes to the pipeline to move the data to a data staging server
        associated with the execution node. Once a job is assigned to a node by
        the remote cluster scheduler, \textit{PegasusLite} is deployed.

        The final execution engine provided by Pegasus is the \textit{Pegasus
        MPI Cluster} (PMC). This execution engine is designed for petascale
        systems for which Pegasus' other strategies leveraging Condor Glideins
        are unsuitable. It employs the Master-Worker architecture, launching a
        single Master and many Workers, and using MPI for communication between
        the Master and Workers. The main limitation of this strategy is that a
        task must use less resources than that available on a single node. Jobs
        are not limited to a single execution engine, and thus, multiple can be
        used in a job to optimize performance. 
 
        As executing a workflow on remote resources can be slow due to job
        queuing times and data transfers, the Mapper will group short-running
        task together to reduce to improve runtime.  When tasks located at the
        same distance from the DAG root have similar runtimes, level-based
        clustering can be used. When runtimes of same-level tasks may vary, it
        may be preferable to use job runtime-based horizontal clustering, to
        achieve load-balancing of clusters. Alternatively, the user may opt for
        label-based clustering, in which the user can define the task clusters
        themselves.

        To avoid running out of disk space during data-intensive workflows, the
        Mapper can add cleanup nodes that remove all unecessary data after a job
        has completed.
 
        \note{Dynamic workflows permitted}

        \subsection{Debugging and provenance} Job monitoring, in Pegasus, is
        achieved through \textit{pegasus-kickstart} (kickstart). Every job
        within a pegasus workflow has a kickstart wrapper that collects runtime
        provenance and performance. Kickstart relays this information to the
        submit host after job completion. This data is ultimately collected and
        stored in a relational database.

        The Mapper Logs contain details on the abstract workflow, the jobs in
        the executable workflow and the mapping of tasks to jobs. Using the
        mapper logs, Pegasus can generate the provenance of the results. Should
        it be required, Pegasus can also generate the provenance of the mapping
        process.

        In addition to the Mapper Logs, there also exists the DAGMan logs,
        generated by HTCondor DAGMan and updated in near real-time. These logs
        capture details on the status of the jobs and their associated pre- and
        post- execute scripts.

        All provenance and debugging data is accessibly through either the
        \textit{pegasus-analyzer} tool or through the provided lightweight web
        interface.  
         
           \section{Make} \subsection{Summary} As the creation of workflow
           activities requires a certain level of programming expertise, Make
           was proposed as a simpler alternative. Make is a tool used for the
           development of directed acyclic dependency graphs. It is available in
           all Linux and UNIX distributions, which relieves the users from
           worflow engine installation. Additionally, Make is capable of
           offering other features necessary to workflow engines, such as,
           reproducibility, parallelization, fault-tolerance and quality
           control. The motivation behind Make is essentially that the tools
           available for simple and efficient pipeline creation already exist
           and there is no need to 'reinvent the wheel'. In addition, the use of
           Make is not limited to the neuroimaging community, and thus support
           for it is widely available. Make has also been adapted for use in
           other programming languages such as SnakeMake for python

 
        \subsection{Pipeline description} Make pipelines are described in
        makefiles, which consist of a set of rules which describe the flow of
        data. Rules defined in make can be define by three components: targets,
        dependecies and recipes. The target is the output (typically file)
        produced by execution of the recipe, whereas the dependecy would be the
        input file from which the target would be created from. The target is
        only modified if its last modified date is early than that of a
        depencies's.  A recipe, in make, is the set of shell commands used to
        create the target from its dependencies. 

In order connect the rules together, it is possible to define "phony" targets.
Phony targets do not correspond to any actual file and will be generated once
all of their dependencies (targets of other rules) have been created. Chaining
multiple rules together will automatically generate the directed acyclic graph
that is the pipeline. This is contrary to other pipelines such as Nipype and
LONI, in which the user must explicitly state all connections \note{as stated in
paper}. Through the use of rules, it is easily possible to reexecute only the
desired parts of the pipeline.

Cleanup after rule execution can be defined using the phony target clean.

To create different targets for the same dependency files organized in different
ways (e.g. subject/timpepoint timepoint/session), it is possible to use symbolic
links for the dependencies, however this may complexify the code. Otherwise, it
may be simpler to copy subject and session makefiles.

        \subsection{Pipeline execution}

        Compared to the previous workflow engines, Make pipeline variables are
        only evaluated at time of use (lazy evaluation). In addition, targets
        are not executed in the order they are specified (similarly to other
        pipeline engines).

Make files are also well-equipped to handle workflows with conditionals. Should
a conditional exist in a workflow, it will be evaluated during the creation of
the DAG (i.e. not at runtime) and only the required components will be executed. 

        As parallelism can be inferred from the dependency graph, parallel
        execution of a make pipeline can occur simply through the addition of
        the -j flag for multicore machines, or by using qmake command to run the
        pipeline on a cluster.        

        By default, make only rebuilds targets whose last modified preceedes
        that of its dependencies. As such, when a task fails and is re-executed,
        only the targets that need to be rebuilt will be rebuilt. It is possible
        to specify reexcution in Make such that the task will be reexecuted in
        case of failure.
 
        \subsection{Debugging and provenance} As users of a preexisting Make
        pipeline may not be aware of the proper usage of each target, it is
        necessary to provide a `help` module that is adaptable to targets being
        added removed or modified. It is possible to define in Make a macro for
        such a purpose that will provide users with usage information when
        requested (e.g. calling make help).

    Debugging in Make can be rather difficult, particularly when executing code
    in parallel. However, pipelines in Make are broken down into many targets
    which can each be tested individually. In addition, Make provides built-in
    flags to aid in debugging. These flags include -n, for printing out the
    commands without any execution and -p to print out all the rules. In the
    instance where pattern substistution is used, -p will also include the
    results obtained from the pattern substitution. More recent versions of GNU
    Make also include the --trace and --output-sync flags which display why each
    recipe was executed and print the parallelized output in sequential order,
    respectively.


    Neuroimaging tools are typically equipped with quality assurance modules. As
    a result, it is possible to utilize the quality assurance modules from these
    tools to provide a complete report for the entire pipeline. The proposed
    solution suggests to write all the quality assurance outputs from each tool
    to a single HTML report with R Markdown or a script. The R Markdown or
    script, with the metrics and images are listed as dependencies to the QA
    report target in Make.


    Only a basic form of data provenance has been implemented by the authors.
    The implemented data provenance consisted of a makefile that calls a bash to
    extract acquisition parameters from the raw input images. The makefile also
    obtains tool versions from the installation directories and pipeline options
    from their respective makefiles. All of these details are stored in a CSV
    file.  The makefile will then call an R Markdown script which will then
    generate an HTML provenance report from the corresponding CSV file.
         
         \section{OpenMOLE} \subsection{summary} OpenMOLE (Open MOdeL
         Experiment) is a domain agnostic pipelining framework that focuses on
         simplifying workflow creation through its Domain Specific Language
         (DSL) built on top of Scala. Through the use of its DSL, much of the
         backend is abstracted from users. It also offers a GUI for users to
         build workflows. Similarly to other existing pipelining frameworks,
         OpenMOLE enables parallelization through plugins, meaning that the
         workflow need not be changed between different distribution
         environments.  Moreover, OpenMOLE supports the deployment of tasks or
         groups of tasks on different computing environments.  Another key
         feature of OpenMOLE is that it attempts to address the reproducibility
         issue through the integration of the CARE archiver. CARE allows for the
         execution of command-line applications, written in any language, and
         attempts to simulate the environment in which the application was
         designed on.  OpenMOLE also provides embedded methods for parameter
         exploration.

        \subsection{Pipeline description} An OpenMOLE workflow consists of tasks
        connected by transitions. Each task contains details such as input,
        output and optional parameters, in addition to embedding the application
        that will be applied to the data. Input values are generally supplied by
        the dataflow through an entry point known as \textit{Sources} and output
        values are returned to the dataflow through \textit{Hooks}.   

      Variables in OpenMOLE can be native types (e.g. int, string), files,
      directories, any Java or Scala types, and user-defined types (i.e. Java or
      Scala classes). The output type of an input supplied to another task
      should be the same as the input type, or be in a child class of the input
      type.

      There exist five types of transitions in OpenMOLE: flat, divergent,
      convergent, explorative and aggregative. A flat transition is one in which
      the output of one task is the input to a single other task. On the other
      hand, a divergent transition is one in which the output of one task serves
      as the input to several other tasks. The inverse of a divergent task is a
      convergent task, in which several tasks supply input to a single task.
      Explorative transitions allow for multiple excutions of the same task on
      the sampling outputs produced by an exploration task. Conversly, an
      aggregative transition gathers all the streams produced by an explorative
      transition. It is also possible to add a stopping condition to terminate
      all streams once a condition is satisfied.

      Tasks are not directly connected to each other by transitions. They are
      encapsulated within \textit{capsules} who are connected to each other by
      transitions. This permits the same task to be used multiple times in the
      workflow by encapsulating them in different capsules.

    There also exists the notion of \textit{Input Slots} in OpenMOLE, which
    enables cycles in the workflow. A task may have several input slots and will
    execute each time that all the transitions belonging to the same input slots
    has passed through.
 
                

        \subsection{Pipeline execution}

      In order for an OpenMOLE workflow to be executed, there must exist a
      single encapsulated task. A task will execute once all its dependencies
      have been met. 

    During execution of a task, the CARETask will decompress the CARE archive
    containing the application and its dependencies and the application is
    deployed within the confines of the decompressed archive. 

OpenMOLE relies on the exit code returned by the application in order to
determine whether to reschedule the task. It is possible to prevent task
rescheduling by setting the boolean flag \textit{errorOnReturnValue} to false.
    
      OpenMOLE workflows can be executed on a variety of platforms such as
      remote servers, clusters and compute grids. As these execution hosts may
      not have all the required dependencies to execute a task, tasks are
      archived using CARE. This ensures that all the required applications and
      their dependencies can be shipped and executed at runtime on any Linux
      host.


        \subsection{Debugging and Provenance} A CARETasks standard output and
        standard error are normally printed out to OpenMOLES's console. For
        distributed executions, this might not be a practical solution, in which
        case such information may be passed through the dataflow using OpenMOLE
        variables.
         
        \section{Nextflow} \subsection{summary} Nextflow is a POSIX-compatible
        pipeline engine for scientific workflows.  It extends the Groovy
        programming language to provide an  intuitive domain-specific language
        (DSL) that enables users to easily describe their workflows. Unlike
        pipeline engines such as Nipype and PSOM, Netflow does not employ DAG.
        It instead adheres to the \textit{Dataflow} programming model which
        decides how data will flow through the pipeline at runtime \note{fix
        this sentence}. Nextflow prides itself on allowing the integration of
        containerized tools to address the issue of numerical instability that
        may arise from different environment setups.
  
        \subsection{Pipeline description} A Nextflow pipeline is represented by
        the flow of data between processes. These processes can be written in
        any Linux-compatible scripting language. As in Nipype, these processes
        are executed in an isolated environment. The only means of communication
        between processes is through asynchronous FIFO queues known as
        \textit{channels}. Each process can contain any number of input and
        output channels. Through the use of channels, input and output can be
        stored not only on disk but also in-memory \note{Nipype can also return
        outputs that are not on disk, so this is nothing special}.  

        As with other pipeline engines (nipype, psom, etc.,), the order in which
        the processes appear is irrelevant. Execution order is determined by the
        dependencies.        

        \subsection{Pipeline execution} The \textit{executor} is responsible for
        the execution of a Nextflow pipeline. By default, it assumes that the
        pipeline is to be run locally. The local executor will use
        multithreading in order to exploit available cores within the local
        environment. To configure the executor to run in a remote distributed
        environment, it is necessary to modify the configuration file to specify
        the environment. By following this approach, Nextflow users can
        seamlessly run their code on various processing platforms with a simple
        modification to a configuration file. In addition to local, Nextflow
        currently has executors for SGE, LSF, SLURM, PBS/Torque, NQSII,
        HTCondor, DRMAA, Ignite, Kubernetes and AWS Batch.

        \subsection{Debugging and Provenance} When executing a pipeline, it is
        always possible to specify the generation of a execution report. The
        report is written in HTML and consists of three sections: Summary,
        Resources and Tasks. Summary provides details on the execution command,
        execution status, overall execution time, in addtion to other metadata.
        The Resources tab, on the other hand, contains several HighChart plots
        representing CPU, memory, time and disk read/write rate of each pipeline
        process. The final section, Tasks, contains many metrics including the
        status and command script of each task. 

    Other reports that can be generated at runtime include the Trace report and
    the Timeline report. The Trace report include details such as CPU usage,
    start time, submission time, etc, of each process that was executed,
    whereas, the Timeline report generates an HTML timeline of all processes
    that have been executed.        

    \note{doesn't seem to be anything on provenance...}

    \section{Galaxy}
        Galaxy is another open-source genomics web-platformed designed to 
        address the
        reproducibility crisis in genomics in addition to facilitating workflow
        creation process for non-programmers and ensuring the transparency of 
        the workflows. It differs from workflow engines in that it is a full
        platform designed for researchers to create and share their workflows
        and data with the community. Building workflows in Galaxy is
        entirely GUI-based enabling users with little to no programming
        knowledge to build workflows. Metadata is extracted from the workflows
        during executions and pipelines can be annoted and shared with other 
        users.

        \subsection{Pipeline description}
        In order to make workflow creation accessible to researchers with little
        -to-no programming skills, Galaxy provides a GUI. 
        The GUI allows users to drag-and-drop CLI tools to the screen to create
        workflows. In order for new tools to be added to Galaxy, the user must 
        write a configuration file which contains details on input and output
        specifications and how to execute the tool. During workflow creation, 
        the workflow editor will determine if the inputs and outputs of 
        connected tools are compatible. 

        Alternatively, users may also create workflows by editing workflows that
        have been published to Galaxy or using their own analysis history to 
        generate a workflow. Any workflow created by a user can be published to
        the Galaxy server and made available to all other users.

        \subsection{Pipeline execution}

        \note{no details on workflow failure}
        As Galaxy is meant to be easy-to-use for researchers, all that is 
        required to execute pipelines is access to their webpage. Users have 
        access to detailed records of the executed workflow's metadata and 
        provenance indicating the outcome of the executed workflow.

        Should users desired to host their own Galaxy server, it is possible to
        install Galaxy locally, on HPC with SLURM or PBS installed or the cloud.

        \subsection{Debugging and provenance} 
        Provenance is a big part of Galaxy as it aims to improve accessibility, 
        reproducibility and transparency of workflows. During workflow execution
        , Galaxy automatically produces metadata for each of the workflow tasks.
        This metadata includes details on the input and output datasets, tools 
        user and parameter values.

        In addition to Galaxy's metadata capture, it is also possible for the 
        user to annotate and tag their workflow. This enables the user to 
        communicate with others the intent of the analysis steps in addition to 
        incorporating search capabilities to their workflows.

        Galaxy also provides the Pages capability. Pages is a web-based document
        that enables users to automatically generate detailed documentation of 
        their analyses.

    \section{Toil}
        \note{obtained from paper's supplementary documentation}
        Toil is another general workflow framework aimed at the processing of 
        scientific big data, with its main focus being on genomics workflows. It 
        prides itself in not only its ability to not only scale the processing 
        of very large datasets, but also its portability, allowing it to be 
        executed on any environment. Toil pipelines can be written in CWL, WDL 
        (Workflow Description Language) and Python. The Toil workflow 
        framework is also open-sourced, making it a community-driven effort. 
        \note{as is the case with most} 

        \subsection{Pipeline description}
           A Toil workflow consists of a DAG of Jobs connected to each other
           through input and output dependencies. All custom user scripts must
           inherit from the Job class. In order to absolve the user from having
           to manually create these classes for each function, Toil provides
           wrapper classes that convert Python functions to Toil jobs directly.
           A Toil job need not be part of the static workflow, it can also be
           added at runtime.

           There exists three types of relationships between Toil jobs: parent, 
           child and follow-on jobs. Child jobs begin execution immediately
           after their Parent completes their execution. Follow-on jobs, however
           , begin execution after the Parent, associated child job and their
           successors complete execution. They are particularly useful as 
           cleanup jobs. Both child jobs and follow-on jobs are executed in
           parallel.

           Variables may be passed at runtime to dynamically added child or
           follow-on jobs, or they can be referenced statically using promises.
           Promises enable the reference to the result to be passed prior to 
           runtime, evaluating the result only at runtime.

           Workflows can be executed or 
           resumed through the Runner class, or alternatively, invoked by the
           Toil class. The latter differing from the Runner class as it serves 
           as the workflow's context manager. The Toil class performs
           a preliminary configuration of the workflow (e.g. file staging).
           All Toil workflows begin with a single job. The value of this root 
           job is returned at the end of the workflow.

            The FileStore class manages the permanent and temporary files
            created by the workflow. This class ensures that all necessary files
            will be cleaned up after job execution, but also that workflows
            can resume from the last successful state. Both temporary 
            directories and temporary files can be created at a user's desired
            location (e.g. local file system, Amazon S3, etc.,). Should files
            and directories need to persist during workflow execution and
            be accessible to all jobs, the FileStore
            can add these files to a global store. All files stored in the 
            global store are immutable, preventing any updates to these files
            from occuring other than deletion. There also does not exist any
            kind of filesystem hierarchy in the global store.\note{allows use
            of various different object stores/ enables caching of files as they
            never change}

            In order to add files to a workflow, it is necessary to stage them
            prior to execution. This is achieved using the \textit{Toil} context
            manager as it is running on the leader prior to execution. It is 
            possible to stage the file as a shared file, if necessary. In the 
            instance of a job failure, the staged file behaves like any other 
            file in the job store. If the file is not cleaned up post execution,
            the file will not need to be staged again during resumption.

            Toil also provides the functionality to startup services, such as
            databases or servers, during pipeline execution. Multiple services
            can be define per job and services can have associated child
            services. This functionality allows, for instance, Apache Spark 
            clusters to be spawned during workflow execution. As services 
            complicate workflow resumption post failure, Toil enables the
            checkpointing of services such that the workflow will resume from a
            checkpoint if any of the successors to the service fail to restart.



        \subsection{Pipeline execution} 
            \note{obtained from website documentation}
            Toil is composed of three components: 1) the Job Store, 2) the Batch
            System and 3) the Provisioner. The Job Store centralizes all the 
            files necessary for the workflow. During a workflow re-execution,
            the job store is used to resume the workflow from its last
            successful task. Currently, there are two types of job stores 
            supported: the file job store and the cloud job store.

            The Batch System specifies which type of scheduler will be used.
            A Toil workflow can either be executed locally, on HPC clusters or 
            the cloud.

            The Provisioner, on the other hand, is used to execute Toil
            workflows on specific cloud platforms. Using the Cluster Utilities
            command line tools, a user can provision nodes for their desired 
            cloud platform.

            Toil executes workflows using a leader-worker architecture. The
            leader is a single-threaded read-only process responsible for 
            determining which jobs are ready to be executed. Workers, on the 
            other hand, are created by the batch-system and are responsible
            for the execution of a job. To alleviate the single-threaded leader
            from having to do an excessive amount of work, the worker maintains
            track of its status (i.e. success or failure) and reports its status
            back to the leader by writing it to the job-store. The worker may
            also choose to immediately execute any child-jobs defined by the 
            job it has executed.

            In the instance when a job or a subset of jobs fail, the workflow
            will continue to execute until it can proceed no more and will throw
            an error. The user is given the option to resolve the bug and resume
            the workflow or delete the job store and reexecute the pipeline 
            entirely.
        \subsection{Debugging and provenance}

        By default, workflow standard input and output is hidden to the user.
        It is however possible to make the log data visible to the user during
        execution or to redirect this data to a logfile.

        Toil also provide a fuctionality, stats, to capture statistics on the 
        workflow in the job store. Examples of the statistics it captures
        includes cpu, memory and job duration.

    \section{Thunder}

        Thunder is a Python-based modular set of tools for the analysis of
        imaging and timeseries data belonging to various domains including
        neuroscience. Although it is not specifically a pipelining framework, it
        can run on top of the Apache Spark framework enabling the Thunder
        pipeline to be extended and customized. By running on top of Spark,
        Thunder leverages its features to improve performance on large scale
        studies. Thunder does not need to be executed on top of a Spark cluster
        and can easily be executed, with the same script, an a local system with
        no Spark installation. 

        \note{The following was extracted from the documenation}
        \subsection{Pipeline description} The API for Thunder, regardless of if
        backed by an Apache Spark cluster or not, remains the same. If not using
        an Spark Resilient Distributed Dataset (RDD), the data can be backed by
        a \textit{numpy} array or a distributed \textit{bolt} array. Thunder is
        broken down into several packages, in which the core package consists of
        tools for reading and writing data, in addition to defining common data
        structures. Users may choose to install other Thunder packages for added
        functionality, or define their own using any Python library.        
 

        Thunder contains two data structures: \textit{images} and
        \textit{series}. As implied by its name, the \textit{images} data type
        may represent a collection of images or volumes.  The \textit{series}
        data type, however, represents a one-dimensional collection that shares
        a common index. Both \textit{images} and \textit{series} can be loaded
        from a list, ndarray or binary file type, however \textit{series} can
        also be loaded from a text file and \textit{images} can be loaded from
        png and tiff files.

        Methods that are common to both data types are, such as \textit{map} and
        \textit{reduce}, are found in the \textit{data} class which is a
        subclass of the \textit{base} class. In order to convert between the
        \textit{images} and \textit{series} data types, the intermediary data
        type \textit{blocks}, child of \textit{base} class,  is used.  Using the
        \textit{blocks} class, it is also possible to convert the data type to
        an ndarray for use with other Python modules. 

        \note{not really anything on execution, debugging and provenance in the
        paper nor in the documentation} \subsection{Pipeline execution} In order
        to use Spark within Thunder, a SparkContext must be started and passed
        to the Thunder API. Should the user need to switch between local and
        Spark frequently, it is possible to use the \textit{station} package as
        a context manager for the various contexts.

        As Thunder is an extension to Spark, numpy and \textit{bolt}, much of
        the execution details are likely differed to the framework that it is
        being executed on.

        \subsection{Debugging and provenance} Thunder does not explicitly
        provide any details on debugging and provenance tracking. Similarly to
        execution, it is expected that much of the debugging and provenance
        details are not obtained from Thunder itself, but rather, from the
        frameworks it is being executed on. For instance, should Thunder be
        exected on a Spark cluster, it is likely that much of the debugging and
        provenance information can be obtained from the Spark logs and RDD
        lineage.
    
\chapter{MapReduce-based BigData frameworks and related components}
\section{mapreduce} Summary of ~\cite{mapred}

        Typically, simple computations need to be performed on increasingly
        growing large datasets. Processing of such large datasets require
        parallelization. As a result, these computations are complicated by
        details of parallelization, fault tolerance, data distribution and load
        balancing. The MapReduce library addresses this issue by allowing the
        expression of simple computations while abstracting the other details.
        The map and reduce paradigm was selected for this purpose as it was
        already commonly used in functional programming languages, such as Lisp,
        and it was observed that many computations could be expressed as such.

        The MapReduce library operates by first dividing the data into
        manageable chunks (16 to 64MB). It then creates multiple copies of the
        program and distributes them across nodes. There exists a single master
        node; the remainder are worker nodes. The master node is responsible for
        delegating map and reduce tasks to the idle workers. There are two data
        structures found in the master. The first stores the state
        (\textit{idle}, \textit{in-progress} or \textit{completed}) of each
        \textit{map} and \textit{reduce} task, and the second stores the
        identity of the non-idle worker nodes. In contrast, the worker nodes are
        responsible for the execution of \textit{map} or \textit{reduce} tasks.
        A worker node assigned a \textit{map} task will parse out the key-value
        pairs from the input file and pass it to the \textit{map} function. The
        intermediate keys produced by the \textit{map} function are buffered in
        memory and periodically written to disk.The locations of these
        intermediate files are sent to the master who is responsible for
        forwarding these locations to the reducer. A worker node assigned a
        \textit{reduce} task uses remote procedure calls to obtain the
        intermediate local files created by the map tasks and sorts the keys
        such that identical keys are grouped together. Each unique key and its
        list of values are then submitted to the \textit{reduce} task and the
        output of the \textit{reduce} task is stored in a final output file.
        Once all \textit{reduce} tasks have completed, the master resumes the
        program and the output is returned to the user.

        In order to ensure fault tolerance, the master regularly pings the
        worker nodes. Should a worker node not respond after a certain delay,
        the master labels the worker as failed. The \textit{map} tasks that were
        completed by the worker are all returned to their original state
        (\textit{idle}) as \textit{map} task data is stored locally. Both the
        completed and in-progress worker's \textit{map} tasks are subsequently
        reassigned to other workers. Only the failed worker's
        \textit{in-progress} \textit{reduce} tasks need to be reassigned.

        The master node undergoes periodic checkpoints, such that the master
        node can be reinitialized from its last checkpoint in case of failure.
        Master node failure is, however, unlikely.

        As network bandwidth is a scarce resource, the master attempts to
        schedule \textit{map} tasks closest to where the data is located. it
        will first attempt to schedule a map task on the host that contains the
        data; should that option not be available, the master will attempt to
        schedule the \textit{map} task on a host nearest to the data (e.g. same
        network switch)

        Ideally, the number of \textit{map} and \textit{reduce} tasks should be
        much larger than the number of available worker nodes.  Having it as
        such improves both dynamic load balancing and recovery time. The number
        of \textit{map} tasks used in practice typically corresponds to the size
        of the input chunks, whereas the number of \textit{reduce} tasks are a
        small multiple of the number of worker nodes used.

        When nearing completion of the MapReduce program, it is possible for
        ``stragglers" (\textit{in-progress} tasks running on nodes with below
        average performance) to delay completion. As a measure to counteract
        this, the master node assigns the same task to idle workers. The task is
        then marked as completed as soon as either the primary or the backups
        complete the task. This strategy has been found to significantly improve
        performance when used with very large datasets.  \note{44\% speedup with
        their sort program}.

        A few extensions have also been added to the library in order to improve
        user experience. An important extension is the introduction of a
        \textit{combiner}, which enables the commutative and associative reduce
        function to be applied to the same keys on the hosts where the
        \textit{map} task was executed. Like the \textit{map} task, the
        \textit{combiner} produces intermediate output which is sent to the
        \textit{reducer}. This alleviates the amount of key-value pairs that
        must be transferred over the network from the map host to the reduce
        host, and as a result, improves performance. Other extensions include
        the ability to create user-defined partitioning functions, a guaranteed
        increasing key sort order, the ability for the user to specify input
        types, an option to skip records that continuously result in execution
        failure, the ability to produce auxiliary output files in \textit{map}
        or \textit{reduce} task, live status updates stream to HTTP server, the
        ability to run all code on a single local machine, and a
        \textit{Counter} class.  

    \section{spark} \begin{itemize} \item based on
        map-reduce \item in-memory processing \item lazy evaluation \item
        general framework \end{itemize} 

    \subsection{Scientific workflows in Spark} \subsubsection{Introduction}
    emphasis on data rather than on interface. transformation describe how
    interfaces interact with the data.  \subsubsection{ariel's paper and
    thunders astronomy - } \subsubsection{Spark on hpc's or } \note{large
    dataset for valduriez} Prior to examining how Big Data frameworks could be
    improved on HPC clusters, it is important to have a baseline of how it
    performs natively on an HPC cluster.  Particularly, how it behave when
    processing scientific workflows. In a recent study done by \cite{valduriez},
    they examined the scalability of Spark on an HPC cluster using a black-box
    scientific workflow. Five tests were performed in their study: 1) Measuring
    execution time when task and resources available increase proportionally, 2)
    Measuring the execution time of very short and short tasks, 3) Measuring the
    execution time of short (5s) to long tasks (120s), 4) Varying the number of
    tasks with fixed task durations, 5) Mixed task durations and varied number
    of tasks. Their results show Spark behaves as expected when tasks and
    resources are increased exponentially, and that Spark scales better than
    expected with tasks of longer durations. This is due to the fact that the
    scheduler is overloaded in the case of many short tasks in addition to their
    being a significant amount of I/O occurring between each task. Although I/O
    is problematic in both data center and HPC infrastructures, HPC is optimized
    for data transfer over the network rather than disk I/O, and therefore, such
    frequent I/O negatively impacts the system. However, many scientific
    workflows are composed of longer tasks, and therefore, Spark scales very
    well with scientific workflows in an HPC environment.  \section{Use of
    containers with Big Data frameworks} \section{Hadoop Distributed file
    system} \note{maybe should go in between mapreduce and spark} 

        The Hadoop Distributed File System (HDFS)\cite{hadoop} is the filesystem
        component of Hadoop. The metadata in HDFS is stored in on a dedicated
        server known as the NameNode, whereas application data is distributed
        across many servers referred to as DataNode. To ensure fault tolerance,
        HDFS replicates the data across multiple DataNodes (default replication
        factor: 3). Not only does this ensure that the data is not lost in the
        event of node failure, but it also increases data transfer bandwidth, as
        the data can be accessed from multiple nodes, and thus, there are more
        opportunities for computations to be performed nearest to where the data
        is located. 

        The NameNode contains the namespace tree, a hierarchy of directories and
        their files, as well as the mapping of the split file blocks to
        DataNodes. The entire namespace is stored in memory. Information on the
        directories and files, such as modification and access times, namespace
        and disk quotes, are stored within \textit{inodes} on the NameNode.  The
        name system's metadata, \textit{inode} and file block mapping, is known
        as the \textit{image}. Persistent record of the image that is stored on
        the local file system are known as \textit{checkpoint}. Locations of
        block replicas are not stored in the checkpoint as they may change over
        time. The \textit{journal} is a log of the modifications made to the
        image. Both the checkpoint and the journal may be copied across servers
        for increased durability. The journal is played back during the NameNode
        restart in order to restore the cluster.

        When a client requests to read data, it must first contact the NameNode.
        The NameNode provides it with the replica locations, and the client
        reads from the DataNode located closest to it. When the client requests
        to write a file, it contacts the NameNode which selects the DataNodes
        that will host the replicas. The client subsequently writes the data
        directly to the DataNodes in a pipeline fashion (The data gets
        propagated through each DataNode by being transferred from the nearest
        DataNode to it).

        There is only one NameNode assigned to each cluster, however, each
        cluster can have multiple clients and execute multiple tasks
        concurrently.


        A block replica stored on a DataNode consists of two files stored on the
        local file system: the metadata and the data. 

        The DataNode connects to the NameNode during startup and performs
        handshake to for DataNode namespace ID and software version
        verification. Should either namespace ID or software version not match
        those of the NameNode, the DataNode shuts down.

        To ensure integrity of the system, a \textit{namespace ID} is assigned
        to all nodes during formatting of the namespace.  DataNodes with
        different namespace IDs to the NameNode will not be permitted to join
        the cluster. Should the DataNode have not been assigned a namespace ID,
        it will be permitted to join the cluster. The cluster's namespace ID
        will then be assigned to that DataNode.


        Following the handshake, the DataNode register with the NameNode using
        their storage ID. The storage ID is a unique ID generated after initial
        registration to the NameNode and that is persistently stored on the
        DataNode. It permits identification of the DataNode in the event of an
        IP address or port change. 
        
        \section{Dask}
        Dask contains three primary abstractions: the \textit{dask.array}, 
        \textit{dask.bag} and \textit{dask.dataframe}. The \textit{dask.array} 
        is an abstraction that enables parallel NumPy-like operations using task
        scheduling. It can additionally be used on data that does not fit in 
        memory. The \textit{dask.array} used blocked array algorithms to break 
        down large commonly-used linear algebra functions into smaller ones that
        can be partially parallelized.

        Dask array functions all create an \textit{Array} object which contains 
        the dask graph \textit{.dask}, details on the overall shape and the 
        shape of each subdivided portion (chunk) of the array (\textit{.chunks})
        , the \textit{.name} representing which keys map to the results, and the
        data type.

        A limitation of the Dask array is that it must know the shape of the 
        array at graph creation time.

        The Dask bag is similar to that of a Python list, however, order cannot 
        be guaranteed. As Dask bags typically operate on Python objects, 
        performance is bound by the Global Interpreter Lock (GIL), therefore 
        multiprocessing schedulers must be used instead of multithreading ones.

        The Dask bag contains functions like \textit{map} and \textit{filter}, 
        in addition to many other functions that can be found in the PyToolz API
        . It is particularly well suited for the handling of data dumps.

        The Dask dataframe consists of a collection of Pandas DataFrames, and 
        thus implements many of the functionalities found in an Pandas DataFrame
        . The dataframe is well suited for running on partioned datasets, where 
        partitions are clearly defined by an index.

        In order to execute the Dask graphs in an efficient manner, proper 
        scheduling is required. Dask uses dynamic scheduling for all their 
        schedulers. When a worker reports completion of a task, the runtime 
        state is updated to reflect that the task has been completed, tasks that
        can now be executed are marked and the data which can be released is 
        determined. To ensure that workers complete execution of related tasks 
        before commencing unrelated ones, a \textit{last in first out} policy is
        used. As Dask graphs are entirely decoupled from the scheduler that they
        use, Dask users are free to design their own scheduler that best suits 
        their needs.


\chapter{Optimizing Big Data frameworks for High-Performance Computing}
	
    \note{precise the systems that will be used. both have distributed memory.
    data centre is hpc cluster with slower network, because may span multiple
    racks and domains} \note{variety of jobs on hpc whereas dedicated map-reduce
    jobs in a data center} \note{talk about spark on cloud and the challenges
    associated with} \note{talk about why this is important} \note{talk }
    High-Performance Computing (HPC) clusters are extensively used by the
    scientific community to process complex, computation intensive problems. Due
    to their importance in the advancement of scientific research, it is
    necessary that Big Data frameworks used to process scientific data are
    compatible with them. 
	
    Big Data frameworks were optimized for a commodity cluster.  Such clusters
    are made up of nodes with homogenous storage connected by lower-end network
    cables (e.g. Ethernet). As a result, moving data from local storage to
    memory was much less costly in these infrastructures than moving data across
    the network. Thus, Big Data frameworks implemented data locality in an
    attempt to reduce costly network (horizontal) transfers and favour more
    performant local storage to memory (vertical) transfers. HPC clusters,
    however, are not necessarily commodity clusters. Large HPC clusters are
    commonly made up of supercomputing nodes connected by a high bandwidth,
    low-latency network (e.g.  InfiniBand). Nodes may contain their own local
    storage (e.g. HDD, SSD, tmpfs) and are all connected to a distributed file
    system (e.g GPFS, Lustre).  Due to their employment of a high bandwidth low
    latency network, network transfers on such an HPC cluster may not be as
    costly as that of a typical data center, and it may in fact be more costly
    to do vertical transfers. Furthermore, HPC systems may enforce a batch
    submission system. This required users to specify number of nodes, cores and
    total processing time; requirements which are unknown to MapReduce users.
	
    This chapter will examine the various efforts by the community to improve
    Big Data framework performance on HPC clusters.
	
    \section{File system improvements}
	

    In paper \note{Scaling Spark on HPC Systems} it was found that Spark on a
    Lustre backend performed significantly (4x) slower than that of a
    workstation using local fast SSDs for storage. This is due to the fact that
    file system metadata latency (more specifically, file open) is the
    determinant factor for performance on HPC, whereas network dominates
    performance in a typical data centre configuration. It was found by the
    authors that using either a local in-memory filesystem or a filesystem
    mounted to a single Lustre file improved performance up to a point where it
    matched that of a workstation. In addition, use of a cache was found to
    reduce file system metadata operations, and thus improve overall
    performance.  The issues related to using an in-memory filesystem is that it
    is limited by the amount of physical memory available, and a job will fail
    if physical memory runs out. Moreover, there is not persistence or
    resilience of data as it is not saved to non-volatile storage (disk). Spark
    fails with medium to large applications due to "lax garbage collection in
    block and shuffle managers". To resolve these issues, the authors attempted
    to make all executors share a descriptor pool. This allowed files that were
    already opened by other executors to be accessed by the executor through the
    descriptor rather than having to reopen it again. However, the node OS image
    is subject to number of Inode constraints as well Lustre limits on number of
    open files for a given job. Due to Inode constraints, a livelock may occur
    when descriptor pool is at capacity and an executor is attempting to open a
    file. The authors also created mounted a local filesystem backed by a Lustre
    file (lustremount) to address the issues related to an in-memory file system
    \note{requires admin privilege}. It was found that the use of an in-memory
    file system improved performance by 7.7x and the lustremount improved
    performance by 6.6x. It was also found that implementation of the filepool
    improved performance.  Another issue observed by the authors is that of poor
    block management as a result of memory constraints resulting in a
    significant amount of vertical movement. When available memory is exhausted,
    the least recently used block is evicted by the block-manager. When a call
    is made to this removed block, it will need to be recomputed, and in the
    process blocks required for recomputation may also be evicted and
    necessitate recomputation of those blocks, thus leading to a chain of
    eviction and recomputation. To resolve this, marking intermediate RDDs as
    persistent, such that their results are saved to storage instead of
    requiring recomputation, may be applied.  \note{they didn't look at
    improving initial application I/O, though believe that it could be improved
    with lustremount and filepool. } \note{for each partition, a shuffle file is
    created and written to as many times as there are partitions. An index file
    is also created which contains shuffle file locations and offsets. Metadata
    access is therefore O(partitions2)}
	
    \subsection{tachyon paper} I/O is a known bottleneck in Spark that is
    exacerbated in HPC systems in which data is stored in a separate cluster
    outside compute nodes. One possible solution used to mitigate the effects of
    the I/O bottleneck is to store data in an in-memory filesystem rather than
    on disk.  Alluxio and Triple-H are two known examples of such filesystems.
	
    Alluxio, formerly known as Tachyon, is an in-memory file system designed to
    improve read and write performance without impacting fault tolerance of the
    system. As write performance is impeded by the need to replicate data,
    Alluxio leverages the concept of lineage to recompute the output if data is
    lost.  Tachyon exhibits 110x faster write throughput than in-memory HDFS and
    4.4x faster end-to-end latency. 

    Alluxio has is made up of two layers: the lineage layer and the persistence
    layer. The lineage layer is the layer that performs recomputations to obtain
    the desired output. The persistence layer contains all the code used to
    regenerate the data, in addition to any data that cannot be recovered
    through lineage as it may be too large to fit in memory or has no lineage
    information stored.
	
    \subsection{triple-H}
	
	
	
	
	
	
	
	
	
	
	
	
	
	
	
    %%%In other words, such frameworks would try to limit data movement, as
    %network cables are expected to have low bandwidth and high latency. In
    %addition, each node in a commodity cluster is expected to have access to
    %its own local storage homogenous to that of the rest of the cluster, thus
    %replication is necessary to ensure data availability. This is, however, not
    %necessarily the case for computing grids. Computing grids may have access
    %to Infinibands, which provide high bandwidth and low latency. Moreover,
    %grids may use heterogenous storage in addition to a parallel file system
    %such as Lustre. As a result, it may, in some cases, be faster to transfer
    %data over the network than to access the nearest copy of the data. As well,
    %data replication may burden a system with centralized storage as multiple
    %copies of the data would be created, taking a significant amount of
    %space.%%%
	
	
	
    %%%Big Data frameworks were not designed with high-performance computing in
    %mind. Big Data filesystems such as were intended to run in data centers
    %with homogenous storage (ex. HDDs, SDDs. etc,.). In such infrastructures,
    %data movement is expensive. To counteract this, strategies were employed to
    %avoid moving data around (data locality). In contrast, HPC clusters use
    %heterogenous storage (ex. a combination of HDD, SSD, Ram Disk and Parallel
    %file systems) with compute nodes connected by high-performance
    %interconnects. HPC nodes may therefore not all be equipped with the same
    %type of storage devices, some of which may be faster than others. Unlike
    %data centers, HPC compute nodes do not have local storage. Moreover, the
    %high-performance interconnects ensure low latency and high throughput of
    %the system. In some instances, it may be even more performant to move data
    %around rather than ensuring data locality. When a single-node HPC cluster
    %using the Luster parallel file system was compared to a workstation with
    %local SSD, it was found that Spark performed about 4x slower on the HPC
    %cluster \cite{Chaimov:2016}.  As HDFS is not conscientious of the
    %underlying infrastructure and expects a data center-type infrastructure, it
    %cannot achieve its full performance potential. %%%
	
    %%%\section{Spark in the cloud}
	
    %%%Using Spark against a typical HPC parallel file system such a Luster has
    %been found to impede scalability. This was found to be due the file system
    %metadata latency, which limits Spark scalability to
    %\textit{O}(10\textsuperscript{2}) cores \cite{Chaimov:2016}.
	
	

    %%%Compute Grids, interconnected networks of computing resources, are
    %heavily used in scientific research for the processing of data. Due to
    %their use and importance to the scientific community, it is necessary that
    %big data frameworks used for processing neuroimaging data are compatible
    %with them. Compute grids typically utilize heterogenous storage, such as a
    %mixture of SSD, HDD, RAM Disk and parallel file systems (e.g. Luster).
    %However, big data frameworks, such as Spark are designed to work
    %efficiently with homogenous storage. Such differences result in reduced
    %performance and inefficient storage use when big data frameworks are used
    %on HPC clusters. This chapter will review the difference techniques applied
    %to big data frameworks as well as HPC and grids to improve collocation of
    %both services.%%%
	
    %%%\subsection{Anatomy of Grid storage} %%Typical grids consist of a cluster
    %of data nodes possibly connected to local storage (e.g. RAM Disk, HDD,
    %SDD), but is also connected through a network to another cluster
    %representing a parallel file system such as Lustre. RAM Disk is the fastest
    %storage as all data is located in memory, whereas SSD provides slightly
    %slower, persistent storage, but is a good alternative for big data. HDDs
    %are the least performant form of local storage.%%% %%Lustre is a stateful,
    %object-based parallel file system. It consists of three components: 1) the
    %meta data server (MDS), 2) the Object Storage Server (OSS) and 3) the
    %Luster Network (LNET), which enables clients to communicate with each
    %other.%%%

	
	
	
    In an attempt to improve performance of HDFS on HPC clusters, Triple-H was
    created.  \section{Why is use of such systems important}
    \section{Scheduling} \subsection{HPC schedulers: SLURM/QSUB} \subsection{Big
    Data schedulers: YARN/MESOS} \subsection{Multi-level scheduling w/ BigData
    schedulers} \chapter{Conclusion}


%%%%%%%%%%%%%%%%%%%%%%%%%%%%%%%%%%%%%%%%%%%%%%%%%%%%%%%%%%%%%%%%%%%%%%%%%%%%%%%
%% Body of Thesis goes here.
%%%%%%%%%%%%%%%%%%%%%%%%%%%%%%%%%%%%%%%%%%%%%%%%%%%%%%%%%%%%%%%%%%%%%%%%%%%%%%%

\addcontentsline{toc}{chapter}{Bibliography} \bibliography{bibliography}
\bibliographystyle{ieeetr}
%\bibliography{abbr,chalin,common,larch,tn}  %place your .bib files here
%\bibliographystyle{alpha}                   %the bibliography style to use

\end{document}
